\begin{problem}[h]
<<echo=FALSE>>=
mass = round(runif(1,1,5))
f = 10*round(runif(1,1,10))
kinetic_friction_coef = 0.1*round(runif(1,1,4))
a = round((f-kinetic_friction_coef*mass*10)/mass,1)

result = a
result_fake1 = mass
result_fake2 = f*kinetic_friction_coef
result_fake3 = kinetic_friction_coef
result_fake4 = mass*f
@
{\bf Newton's Laws with friction.} A block of $\Sexpr{(mass)}$ kg is moving on a horizontal surface due to a force $F$ of $\Sexpr{(f)}$ N parallel to that surface. If the kinetic friction coeficient between the block and the surface is $\Sexpr{(kinetic_friction_coef)}$, the aceleration of the block should be...
\begin{answers}{3}
    \bChoices[random]
    \Ans1\label{resp4.2} \Sexpr{(result)}$\frac{m}{s^2}$\eAns
    \Ans0 \Sexpr{(result_fake1)}$\frac{m}{s^2}$\eAns 
    \Ans0 \Sexpr{(result_fake2)}$\frac{m}{s^2}$\eAns
    \Ans0 \Sexpr{(result_fake3)}$\frac{m}{s^2}$\eAns
    \Ans0 \Sexpr{(result_fake4)}$\frac{m}{s^2}$\eAns
    \eFreeze
    \Ans0 None of them\eAns
    \eChoices 
\end{answers}
\begin{solution}
Using Second's Newton Law on the vertical axis,\\
$N = F_g = mg$\\
Then, using First's Newton Law on the horizontal axis and isolating the aceleration,\\
$a_x = \frac{\Sigma F_x}{m} = \frac{F-f_k}{m} = \frac{F-\mu\cdot N}{m} =\frac{F-\mu\cdot m \cdot g}{m} = \frac{\Sexpr{(f)}N-\Sexpr{(kinetic_friction_coef)}\cdot\Sexpr{(mass)}kg\cdot 10m/s^2}{\Sexpr{(mass)}kg} = \Sexpr{(a)}m/s^2$.\\
Therefore, the right answer is letter \textbf{(\REF*{resp4.2})}.
\end{solution}
\end{problem}