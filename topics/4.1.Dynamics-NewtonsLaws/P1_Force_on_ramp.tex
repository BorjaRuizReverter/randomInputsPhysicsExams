\begin{problem}[h]
<<echo=FALSE>>=
weight = 10*round(runif(1,1,10))
ang_grad = 15*round(runif(1,2,4))
if(ang_grad==30){
  denominator_cos_ang=1 
  denominator_sen_ang=3}
if(ang_grad==45){
  denominator_cos_ang=2
  denominator_sen_ang=2}
if(ang_grad==60){
  denominator_cos_ang=3
  denominator_sen_ang=1}

result = weight/2
result_fake1 = weight
result_fake2 = weight*2
@
{\bf Newton's Laws.} A block that weights $\Sexpr{(weight)}$N is being pushed over a ramp inclined $\Sexpr{(ang_grad)}$\textdegree due to a force $F$, which is parallel to the ramp. The movement is straight and uniform and without friction. What is the module of that force $F$?
\begin{answers}{3}
    \bChoices[random] 
    \Ans1\label{resp4.1}\Sexpr{(result)}$\sqrt{\Sexpr{(denominator_sen_ang)}}$N\eAns
    \Ans0 \Sexpr{(result_fake1)}N\eAns
    \Ans0 \Sexpr{(result_fake2)}$\sqrt{\Sexpr{(denominator_sen_ang)}}$N\eAns
    \Ans0 \Sexpr{(result)}N\eAns
    \Ans0 \Sexpr{(result_fake1)}$\sqrt{\Sexpr{(denominator_sen_ang)}}$N\eAns
    \eFreeze
    \Ans0 None of them\eAns
    \eChoices
\end{answers}
\begin{solution}
Choosing a reference with the horizontal axe $x$ parallel to the ramp and using First Newton's Laws on the horizontal:\\
$\Sigma F{x} = 0$.\\
Therefore,\\
$F = P_x = P \sin \theta = \Sexpr{(weight)}N\sin \Sexpr{(ang_grad)}^{\circ} = \Sexpr{(weight/2)}\sqrt{\Sexpr{(denominator_sen_ang)}}N$.\\
Which means that the right answer is the letter \textbf{(\REF*{resp4.1})}.
\end{solution}
\end{problem}